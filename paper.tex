\documentclass[12pt]{article}
\usepackage[utf8]{inputenc}
\usepackage{amsmath}
\usepackage{diffcoeff}
\usepackage{pgf}

\usepackage[backend=bibtex,style=nature,hyperref=true,autocite=superscript]{biblatex}
\addbibresource{paper.bib}

\usepackage{hyperref}
\hypersetup{
  colorlinks = true,
  citecolor = {blue}
}

%\usepackage{layouts}

\newcommand{\abs}[1]{\left | #1 \right |}
\newcommand\numberthis{\addtocounter{equation}{1}\tag{\theequation}}

\title{Modeling of COVID-19 death cases in Spain by Differential Equation}
\author{Bao-Hiep Le}
\date{\today}

\begin{document}

\maketitle

% Set this value as figure width in matplotlib
%\printinunitsof{in}\prntlen{\textwidth}

This paper uses data from Our World in Data COVID-19 dataset \autocite{owid-covid-data}.
\section*{Introduction}

\begin{align*}
\frac{1}{P} \cdot \diff{P}{t} &= aP + b
\end{align*}

\begin{equation} \label{int_eqn}
\int \frac{\dl{P}}{P(aP + b)} = \int \dl{t}
\end{equation}

To evaluate the integral on the left side, we use partial fraction decomposition:

\begin{align*}
\frac{1}{P(aP + b)} = \frac{X}{P} + \frac{Y}{aP + b}
\end{align*}

Multiplying both sides of this equation by $P(aP + b)$ and rearranging the terms, we have:

\begin{align*}
1 = (Xa + Y)P + Xb
\end{align*}

The polynomials are identical, so their coefficients must be equal:

\begin{align*}
Xa + Y &= 0 \\
Xb &= 1
\end{align*}

We get $X = \dfrac{1}{b}$ and $Y = \dfrac{-a}{b}$. Therefore:

\begin{align*}
\frac{1}{P(aP + b)} = \frac{1}{b} \cdot \frac{1}{P} - \frac{a}{b} \cdot \frac{1}{aP + b}
\end{align*}

We can now rewrite equation \ref{int_eqn} as:

\begin{align*}
\int \left ( \frac{1}{b} \cdot \frac{1}{P} - \frac{a}{b} \cdot \frac{1}{aP + b} \right ) \dl{P} &= \int \dl{t} \\
\frac{1}{b} \ln \abs{P} - \frac{1}{b} \ln \abs{aP + b} &= t + C \\
\ln \abs{\frac{P}{aP + b}} &= bt + bC \\
\abs{\frac{P}{aP + b}} &= e^{bt + bC} = e^{bt}e^{bC} \\
\frac{P}{aP + b} &= Ae^{bt} \numberthis \label{int2_eqn}
\end{align*}

where $A = \pm e^{bC}$. Solving for P, we get:

\begin{align*}
P = \frac{bAe^{bt}}{1 - aAe^{bt}}
\end{align*}

We find the value of A by putting $t = 0$ in equation \ref{int2_eqn}:

\begin{align*}
A = \frac{P_0}{aP_0 + b}
\end{align*}

\begin{align*}
\diff{P}{t} &= \dfrac{b^2Ae^{bt}}{\left ( 1 - aAe^{bt} \right ) ^2}
\end{align*}

\printbibliography

\end{document}
